%   ABSTRACT

An understanding of polycrystalline materials, ranging from alloys to certain ceramics and polymers and beyond, is of great importance for modern society. These materials typically form through the process of nucleation, a thermally activated phase transition. The numerical modeling of this phase transition is problematic for traditional numerical techniques: the commonly used phase field methods' resolution does not extend to the atomic scales at which nucleation takes places, while atomistic methods such as Molecular Dynamics are incapable of scaling to the mesoscale regime where late-stage growth and structure formation takes place following earlier nucleation. As such, there is interest in examining whether the Phase Field Crystal (PFC) model, which attempts to bridge the atomic and mesoscale regimes, is capable of modeling nucleation. In this work, we numerically calculate nucleation rates and incubation times in the PFC model. We show qualitative agreement with classical nucleation theory (CNT), a single-variable stochastic model. We also examine the form and behavior of nuclei at early formation times, finding disagreement with some basic assumptions of CNT. We then argue that a quantitatively correct nucleation theory for the PFC model would require extending CNT to a multi-variable theory.